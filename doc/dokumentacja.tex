\documentclass{article}

\nonstopmode

\usepackage{makecell}
\usepackage{polski}
\usepackage[margin=1in]{geometry}
\usepackage{multicol}
\usepackage{graphicx}
\usepackage{float}
\usepackage{verbatim}
\usepackage[shortlabels]{enumitem}
\usepackage{color}
\usepackage{tikz}
\usepackage{xcolor}
\usepackage{listings}
\usepackage{caption}
\DeclareCaptionFont{white}{\color{white}}
\DeclareCaptionFormat{listing}{\colorbox{violet}{\parbox{\textwidth}{#1#2#3}}}
\captionsetup[lstlisting]{format=listing,labelfont=white,textfont=white}

\lstset{
    literate={ą}{{\k{a}}}1
             {ć}{{\'c}}1
             {ę}{{\k{e}}}1
             {ł}{{\l{}}}1
             {ń}{{\'n}}1
             {ó}{{\'o}}1
             {ś}{{\'s}}1
             {ź}{{\'z}}1
             {ż}{{\.z}}1
}

\begin{document}

\begin{titlepage}
    \AddToShipoutPicture*{\BackgroundPic}

   \begin{center}
       \textbf{Produktywny chatbot Rasa z discordem}

       \textbf{Dokumentacja}


        \vspace{5cm}  
        
        \begin{figure}[htp]

        \centering
        \includegraphics[width=.3\textwidth]{figures/discord-mark-blue.png}\hfill
        \includegraphics[width=.1\textwidth]{figures/plus-symbol-button.png}\hfill
        \includegraphics[width=.3\textwidth]{figures/idqvezu_8-_1716542229027.png}


        \end{figure}     
     
            
        \vspace{8cm}  

       Wydział Informatyki Politechniki Białostockiej\\
       Politechnika Białostocka\\
       Data:
            
   \end{center}
\end{titlepage}

\tableofcontents

\section{Wprowadzenie i teoria}
Rasa to framework, który pozwala na tworzenie chatbotów używających ai. Dzięki
niemu można prowadzić konwersację z botem używając różnych API oraz modeli. Aby
w pełni zrozumieć czym jest Rasa trzeba wpierw dowiedzieć się czym są modele
uczenia maszynowego, CDD oraz NLU.

\subsection{Model uczenia maszynowego}
Jest to obiekt używający różnych algorytmów do znalezienia wzorców w podanych
danych treningowych i na ich podstawie stara się przewidzieć wynik. Model ten
może używać różnych algorytmów. Takich jak:
\begin{itemize}
    \item[\textcolor{violet}{\textbullet}] Regresja Liniowa
    \item[\textcolor{violet}{\textbullet}] Maszyna Wektorów Nośnych
    \item[\textcolor{violet}{\textbullet}] Sieci neuronowe
\end{itemize}

\subsection{NLU (Natural Language Understanding)}
Jest to część rasy, która odpowiada za analizę poszczególnych wiadomości od 
użytkownika, czyli klasyfikacja intencji, wyodrębnienie zmiennych z treści
wiadomości oraz analiza odpowiedzi. Głównym komponentem są \emph{intencje},
które mogą zawierać \emph{byty}.

\subsubsection{Intencje (ang. Intents)}
Są to przykładowe wiadomości, jakie użytkownik może wprowadzić. Im mamy ich
więcej, tym mniejsza jest szansa, że bot się pomyli. Przykładowa intencja
witania się w Rasa wygląda następująco:

\begin{lstlisting}[caption=Przykładowa intencja]
- intent: przywitanie
  examples: |
    - "Cześć!"
    - "Siema"
    - "Hej"
    - "Dzień dobry"
\end{lstlisting}

\subsubsection{Byty (ang. Entities)}
Intencje mogą też zawierać byty, czyli informacje zawarte w treści wiadomości.
Aby móc je wydobyć potrzebne są specjalne dane treningowe oraz
Extractor(ustawiany w konfiguracji chatbota). Takie informacje mogą być
przydatne, kiedy chcemy, żeby użytkownik w wiadomości podawał na dane, które
możemy później wykorzystać. Przykładem użycia bytów może być firma, która
zajmuje się klawiaturami i do tego potrzebuje wiedzieć, jaki model klawiatury
jest używany przez użytkownika.

\begin{lstlisting}[caption=Przykładowa intencja z użyciem bytów]
- intent: 
  examples: |
    - "Zepsułem [superKlawiatura3000]
    - "Nie działa mi [średniaKlawiatura1a00d0]
    - "co mi powiesz o modelu [MFA8W221Ad]?
\end{lstlisting}

\subsection{CDD (Conversation-Driven Development)}
CDD jest procesem polegającym na słuchaniu użytkowników i na podstawie
zebranych informacji ulepszaniu chatbota. Jest to szczególnie ważny proces,
ponieważ nie da się przewidzieć, co powie użytkownik. Użytkownicy używają
swoich własnych stwierdzeń oraz wyrażeń, które mogą okazać się niespodziewane.
Proces ten nie jest liniowy, ale można go podzielić na pewne kroki:
\begin{enumerate}
    \item[\textcolor{violet}{\textbullet}] Udostępnienie informacji użytkownikom,
    \item[\textcolor{violet}{\textbullet}] Regularne sprawdzanie oraz ocenianie wyników konwersacji,
    \item[\textcolor{violet}{\textbullet}] Zapisywanie niespodziewanych wiadomości i dodawanie ich do danych
    uczących,
    \item[\textcolor{violet}{\textbullet}] Testowanie asystenta(chatbota),
    \item[\textcolor{violet}{\textbullet}] Zapisywanie błędów oraz mierzenie wydajności,
    \item[\textcolor{violet}{\textbullet}] Naprawa niepoprawnych konwersacji.
\end{enumerate}

\subsection{Dane konwersacji}
Po napisaniu intencji bot musi wiedzieć, co ma odpowiedzieć i do tego będą
potrzebne \emph{scenariusze (ang. stories)} oraz \emph{zasady (ang. rules)}.
Użytkownik może obrać różne ścieżki, a niemożliwym jest zapisanie i
nauczenie go wszystkich możliwości, dlatego będziemy wykorzystywać proces CDD.

\subsubsection{Scenariusze}
Scenariusze to rozmowy, w których definiuje się, co bot ma odpowiadać na zadane
mu pytania. Składają się one z intencji, bytów oraz akcji. Akcja jest
odpowiedzią na daną intencję, może być ona prostym zdaniem, albo można
zdefiniować skrypt w języku Python. W historiach możemy definiować różne
ściężki, dzięki czemu są elastyczne. Przykładem może być zwykła rozmowa.

\begin{lstlisting}[caption=Przykładowa historia]
- story: zwykła rozmowa
  steps:
   - intent: przywitanie
   - action: utter_zapytanie_co_u_ciebie
   - intent: złe_samopoczucie
   - action: utter_pocieszenie
\end{lstlisting}

\begin{lstlisting}[caption= Jak zdefiniowane są odpowiedzi]
utter_zapytanie_co_u_ciebie:
- text: "Cześć, co u ciebie?"
\end{lstlisting}

\subsubsection{Zasady}
Zasady, to część to rozmowy, która nigdy nie może zmienić toru. W odróżnieniu
od scenariuszy, w zasadach nie ma różnych ścieżek. Dana zasada jest definitywna
i jeśli bot odbierze jakąś intencję, to zasada zmusi bota, żeby zawsze na nią
odpowiadał w ten sam sposób. Zasad nie można pisać jak scenariuszy, ponieważ
można dodać tylko jedną interakcję użytkownika.

\begin{lstlisting}[caption=Przykładowa zasada]
- rule: Zawsze odpowiadaj na przywitanie
   - intent: przywitanie
   - action: utter_zapytanie_co_u_ciebie
\end{lstlisting}

\begin{lstlisting}[caption= Jak zdefiniowane są odpowiedzi]
utter_zapytanie_co_u_ciebie:
- text: "Cześć, co u ciebie?"
\end{lstlisting}

\subsection{Rasa}
Teraz, kiedy znane są potrzebne pojęcia, to można odpowiedzieć dokładniej, czym
jest Rasa? Jest to framework, który używa wielu modeli uczenia maszynowego oraz
technik NLP (Natural Language Processing). Modele są z góry skonfigurowane, ale
jest możliwość dostosowania konfiguracji do własnych potrzeb. Rasa pozwala na
definiowanie intencji (ang. intents), które reprezentują zamiary użytkowników
wyrażane w formie wypowiedzi. Intencje są kluczowe dla zrozumienia, czego
użytkownik oczekuje od systemu. Dodatkowo, framework wspiera pracę z bytami
(ang. entities), które są istotnymi informacjami wyekstrahowanymi z wypowiedzi
użytkownika, takimi jak nazwy, daty, liczby itp. Rasa działa zgodnie z
podejściem Conversation-Driven Development (CDD), które promuje iteracyjne
rozwijanie systemów konwersacyjnych na podstawie rzeczywistych interakcji z
użytkownikami. Dzięki temu, scenariusze (ang. stories) mogą być tworzone i
modyfikowane w celu lepszego odwzorowania rzeczywistych dialogów. Framework
umożliwia także tworzenie zaawansowanych scenariuszy dialogowych, które
definiują sekwencje interakcji między użytkownikiem a botem. Te scenariusze
pomagają w zarządzaniu przepływem konwersacji i zapewniają, że bot reaguje w
odpowiedni sposób na różne sytuacje. Rasa wspiera również zasady (ang. rules),
które pozwalają na definiowanie specyficznych reakcji bota w określonych
sytuacjach. Zasady mogą być używane do obsługi prostych przypadków użycia lub
do implementacji logiki, która nie wymaga zaawansowanego uczenia maszynowego.
Podsumowując, Rasa to potężne narzędzie dla tworzenia inteligentnych asystentów
konwersacyjnych, które łączy zaawansowane techniki NLP, elastyczność w
konfiguracji modeli oraz podejście CDD do ciągłego ulepszania systemu na
podstawie rzeczywistych interakcji użytkowników.

\section{Jak używać?}
Pierwszym krokiem jest wytrenowanie bota za pomocą komendy:
\verb|rasa train|
Kolejnym krokiem jest ustawienie bota na discordzie, do niego trzeba stworzyć
token bota discordowego w serwisie Discord Developer Console. Następnie trzeba
użyć komendy:
\verb|echo "DISCORD_TOKEN=..." > discord_bot/.env|
co pozwoli na dodanie zarządzanie botem z discorda.
Następnie trzeba włączyć rasę oraz połączyć ją z discordem:
\begin{enumerate}
    \item[\textcolor{violet}{\textbullet}] W pierwszym terminalu należy udostępnić api rasy: 
        \verb|rasa run --enable-api|.
    \item[\textcolor{violet}{\textbullet}] W drugim terminalu należy włączyć serwer własnych akcji:
        \verb|rasa run actions|.
    \item[\textcolor{violet}{\textbullet}] W trzecim terminalu trzeba włączyć bota na discordzie za pomocą
        komendy: \verb|python discord_bot/bot.py|.
\end{enumerate}
Alternatywnie można używać bota nie w discordzie, tylko w terminalu. Przy
wybraniu tej opcji nie da się używać funkcjonalności Pomodoro.
\begin{enumerate}
    \item[\textcolor{violet}{\textbullet}] Najpierw należy włączyć serwer własnych akcji w osobnym terminalu:
        \verb|rasa run actions|.
    \item[\textcolor{violet}{\textbullet}] Potem, żeby aktywować bota w terminalu, trzeba użyć komendy: 
        \verb|rasa shell|.
\end{enumerate}
\section{Projekt produktywny chat}
\subsection{Wstęp}
W pliku domain występują podstawowe akcje typu response:
\begin{itemize}
    \item[\textcolor{violet}{\textbullet}] \verb|utter_greet| - "Cześć, co chcesz dzisiaj robić?"
  \item[\textcolor{violet}{\textbullet}] \verb|utter_goodbye| - "Mam nadzieję, że Ci pomogłem."
  \item[\textcolor{violet}{\textbullet}] \verb|utter_thank| - "Nie ma problemu, możesz zawsze na mnie liczyć"
  \item[\textcolor{violet}{\textbullet}] \verb|utter_get_task| - "Jakie zadanie chciałbyś dodać?"
  \item[\textcolor{violet}{\textbullet}] \verb|utter_get_task_to_remove| - "Jakie zadanie chciałbyś usunąć?"
  \item[\textcolor{violet}{\textbullet}] \verb|utter_suggest_pomodoro| - "Metoda Pomodoro polega na 25-minutowych blokach pracy z krótkimi przerwami. Chcesz ją wypróbować?"
  \item[\textcolor{violet}{\textbullet}] \verb|utter_pomodoro_ready| - "Czy jesteś na to gotowy? Jeśli tak, to odpalę pomodoro, ale nie martw się, przypomnę ci o rozpoczęciu kolejnego cyklu lub przerwy."
  \item[\textcolor{violet}{\textbullet}] \verb|utter_pomodoro_start| - "INITIATE\_POMODORO"
  \item[\textcolor{violet}{\textbullet}] \verb|utter_deny| - "Może innym razem"
  \item[\textcolor{violet}{\textbullet}] \verb|utter_laptop_greet| - "Witaj! Czy mogę Ci pomóc z wyborem laptopa?"
  \item[\textcolor{violet}{\textbullet}] \verb|utter_ask_gaming| - "Czy laptop ma być używany do gier?"
  \item[\textcolor{violet}{\textbullet}] \verb|utter_ask_portable| - "Czy ważna jest dla Ciebie lekkość i przenośność?"
  \item[\textcolor{violet}{\textbullet}] \verb|utter_ask_budget}| - "Czy masz ograniczony budżet?"
  \item[\textcolor{violet}{\textbullet}] \verb|utter_recommend_gaming_laptop| - "Polecam laptop do gier. Są one potężniejsze, ale też droższe i cięższe."
  \item[\textcolor{violet}{\textbullet}] \verb|utter_recommend_portable_laptop| - "Polecam ultrabooka. Są lekkie i przenośne, idealne do pracy w ruchu."
  \item[\textcolor{violet}{\textbullet}] \verb|utter_recommend_budget_laptop| - "Polecam laptop budżetowy. Są przystępne cenowo, ale mogą mieć ograniczoną moc."
  \item[\textcolor{violet}{\textbullet}] \verb|utter_recommend_general_laptop| - "Wydaje się, że nie masz szczególnych wymagań. Zalecam uniwersalny laptop."
  \item[\textcolor{violet}{\textbullet}] \verb|utter_default| - "Nie rozumiem, czy mógłbyś to powtórzyć?"
  \item[\textcolor{violet}{\textbullet}] \verb|utter_functions| - Potrafie bardzo wiele rzeczy, ale możesz się skupić na:
    \begin{itemize}
      \item Zarządzanie zadaniami (dodawanie, usuwanie i oczywiście pokazanie)
      \item Mogę cię zmotywować
      \item Możemy też popracować z pomocą techniki pomodoro
      \item Chętnie ci pomogę w kupnie laptopa
      \item (szept) mogę też ci coś powiedzieć o pokemonach
    \end{itemize}
\end{itemize}
Każda intencja ma co najmniej 40 przykładów. Podstawowe intencje:
\begin{itemize}
    \item[\textcolor{violet}{\textbullet}] \verb|greet| - Przywitanie.
    \item[\textcolor{violet}{\textbullet}] \verb|thank|- Podziękowanie.
    \item[\textcolor{violet}{\textbullet}] \verb|inform| - Prośba o podanie funkcjonalności.
    \item[\textcolor{violet}{\textbullet}] \verb|goodbye| - Pożegnanie.
    \item[\textcolor{violet}{\textbullet}] \verb|confirm| - Potwierdzenie.
    \item[\textcolor{violet}{\textbullet}] \verb|deny| - Zaprzeczenie.
\end{itemize}
W pliku sotries można znaleźć podstawowe historie:
\begin{itemize}
    \item[\textcolor{violet}{\textbullet}] \verb|story_functions| - Podanie listy funkcjonalności.
    \item[\textcolor{violet}{\textbullet}] \verb|st_greet| - Przywitanie się.
    \item[\textcolor{violet}{\textbullet}] \verb|st_goodbye| - Pożegnanie się.
    \item[\textcolor{violet}{\textbullet}] \verb|thank| - Odpowiedź na podziękowanie.
\end{itemize}

\subsection{Konfiguracja}
Konfiguracja jest ustawiona po SpacyNLP, ponieważ ten projekt obsługuje
odpowiedzi, które mogą być z różnych dziedzin, a Spacy jest dużo lepiej
dostosowane do takiego działania niż podstawowa konfiguracja. Do Spacy użyto
modelu języka polskiego \verb|pl_core_news_sm|. Dodatkowo w konfiguracji
zmieniono zasady. Podstawową zmianą było dodanie wskaźnika pewności siebie,
który jest ustawony na 60\%. Jeśli bot nie jest pewny swojego wyboru, to wyśle
odpowiedź, że nie rozumie wiadomości. Dodatkowo zwiększono liczbę epok w obu
UnexpecTEDIntentPolicy(tutaj też zmieniono rozmiar wsadów) i TEDPolicy.

\subsection{Zarządzanie zadaniami}
Chatbot może dodawać, usuwać oraz pokazywać zadania. Każdy użytkownik na kanale
discorda ma oddzielną listę, te listy oraz ich właściciele są zapisane w pliku
tasks.json. Chatbot czyta wiadomości wysłane z discorda i dzięki tej
komunikacji może odczytać informację o nadawcy wiadomości. To pozwala na
przypisanie oddzielnej listy zadań dla każdego użytkownika. Są tutaj używane
trzy akcje niestandardowe:

\begin{itemize}
    \item[\textcolor{violet}{\textbullet}] \verb|action_add_task| - Akcja odczytuje listę zadań, a jeśli
        użytkownik jeszcze nic nie dodał, to tworzy nowego użytkownika.
        Następnie dodaje nowe zadanie na koniec listy.
    \item[\textcolor{violet}{\textbullet}] \verb|action_remove_task| - Akcja odczytuje listę zadań, a jeśli
        użytkownik jeszcze nic nie dodał, wyświetla odpowiedni komunikat.
        W przeciwnym przypadku przeszukuje listę, a następnie usuwa zadanie
        z listy.
    \item[\textcolor{violet}{\textbullet}] \verb|action_list_tasks| - Akcja odczytuje listę zadań, a jeśli
        użytkownik jeszcze nic nie dodał, to wyświetla odpowiedni komunikat.
        W przeciwnym przypadku iteruje przez listę i wypisudodanie zadania.je kolejno zadania.
\end{itemize}
Przykładowa akcja:
\begin{lstlisting}[language=Python, caption=Akcja dodawania]
class ActionAddTask(Action):
    def name(self) -> Text:
        return "action_add_task"

    def run(self, dispatcher: CollectingDispatcher,
            tracker: Tracker,
            domain: Dict[Text, Any]) -> List[Dict[Text, Any]]:

        # Przygotowanie danych do pliku.
        path = Path("./data/tasks.json")
        metadata = tracker.latest_message.get("metadata", {})
        user_name = metadata.get("user_name", "nieznany użytkownik")
    
        # Sprawdzenie, czy plik istnieje.
        if not path.exists():
            with open(path, "w") as file:
                json.dump({}, file)
        
        # Otwarcie pliku.
        with open(path, "r") as file:
            try:
                data = json.load(file)
            except json.JSONDecodeError:
                data = {}
        
        # Pobranie ostatnio wysłanej wiadomości przez użytkowika.
        task = tracker.latest_message['text']

        # Dodanie zadania do tablicy z zadaniami.
        if user_name not in data:
            data[user_name] = {"tasks": [{"task": task}]}
        else:
            data[user_name]["tasks"].append({"task": task})
        
        # Zaktualizowanie pliku nową listą.
        with open(path, "w") as file:
            json.dump(data, file, ensure_ascii=False, indent=4)
        
        # Powiadomienie użytkownika o dodanym zadaniu.
        dispatcher.utter_message(text=f"Dodano zadanie: {task}")

        return []
\end{lstlisting}

Ta funkcjonalność wykorzystuje intencje:
\begin{itemize}
    \item[\textcolor{violet}{\textbullet}] \verb|add_task| - Prośba o dodanie zadania.
    \item[\textcolor{violet}{\textbullet}] \verb|delete_task| - Prośba o usunięcie zadania.
    \item[\textcolor{violet}{\textbullet}] \verb|give_task| - Przykładowe zadania, jakie może podać użytkownik.
        Jest tam 210 przykładów poczynając od prostych takich jak: "Kup mleko",
        a kończąc na bardziej abstrakcyjnych: "Wysłuchaj ciszy".
    \item[\textcolor{violet}{\textbullet}] \verb|list_tasks| - Prośba o wypisanie zadań.
\end{itemize}
Dodano tutaj trzy proste scenariusze, które połączono z zasadami
zabezpieczającymi, żeby po prośbie o usunięcie/dodanie zadania, zawsze było
można je podać.
Scenariusze:
\begin{itemize}
    \item[\textcolor{violet}{\textbullet}] \verb|Adding a task|
    \item[\textcolor{violet}{\textbullet}] \verb|Remove a task|
    \item[\textcolor{violet}{\textbullet}] \verb|Listing tasks| 
\end{itemize}
Zasady:
\begin{itemize}
    \item[\textcolor{violet}{\textbullet}] \verb|task_remove|
    \item[\textcolor{violet}{\textbullet}] \verb|task_add|
\end{itemize}


\subsection{Cytaty motywacyjne}
Kolejną funkcjonalnością chatu jest wysyłanie motywujących cytatów, kiedy
użytkownik np. czuje się przytłoczony lub traci wenę do pracy. Za działanie
odpowiada jedna niestandardowa akcja(\verb|action_get_quote|). Wysyła ona
zapytanie do zewnętrznego api i w przypadku powodzenia wypisuje cytat wraz z
jego autorem, natomiast w przeciwnym przypaku wypisuje odpowiednie
powiadomienie. W tej funkcjonalności występuje tylko jedna intencja z prośbą o
zmotywowanie(\verb|offer_motivation|) oraz jeden scenariusz(Motivational quote
generator) łączący tą akcję z niestandardową akcją.


\subsection{Informacje o Pokémon'ach}
W tej funkcjonalności działanie jest takie same jak w poprzedniej, z tą
zmianą, że trzeba pobrać więcej informacji o Pokémon'ie:
nazwa, wzrost, waga, typ. Występuje tutaj jedna, ale bardzo obszerna intencja 
\verb|get_pokemon| zawierająca 150 przykładów, duża ich ilość jest spowodowana
ogromną ilością istniejących Pokémon'ów(w dniu pisania tej dokumentacji liczba
ta wynosi 1025). Scenariusz to \verb|Pokemon|. Do zbierania informacji, jakiego
pokemona chce poznać użytkownik, wykorzystano byt \verb|pokemon|, który jest
zapisywany w slocie \verb|pokemon|.

\subsection{Doradca w zakupie laptopa}
W tej funkcjonalności chatbot zostaje doradcą zakupowym, który zadaje kilka
pytań, na które można odpowiedzieć twierdząco lub przecząco. Na podstawie
odpowiedzi rekomenduje odpowiedni rodzaj laptopa. Konwersację zaczyna prośba
użytkownika o rekomendację laptopa (\verb|laptop_greet|) Jest to rozmowa, która
pokazuje jak rasa pozwala na wybranie różnych możliwych ścieżek wyboru w
konwersacji. Wykorzystano tutaj kilka scenariuszy, co tworzy kilka różnych
ścieżek:
\begin{itemize}
    \item[\textcolor{violet}{\textbullet}] \verb|Laptop Recommendation| - Podstawowa ścieżka, w której
        użytkownik od razu zgadza się na wybór laptopa gamingowego i nie ma
        sensu dalej prowadzić konwersacji.
    \item[\textcolor{violet}{\textbullet}] \verb|Laptop Recommendation - Portable| - Ścieżka, w której
        użytkownik wybiera laptop przenośny i lekki.
    \item[\textcolor{violet}{\textbullet}] \verb|Laptop Recommendation - Budget| -- Ścieżka, w której
        użytkownik wybiera laptop budżetowy. 
    \item[\textcolor{violet}{\textbullet}] \verb|Laptop Recommendation - General| -- Ścieżka, w której
        użytkownik nie jest pewnien, więc jest mu zarekomendowany laptop
        ogólny. 
\end{itemize}
Dodatkowo potrzebne są odpowiednie zasady, które ograniczają możliwości
wyboru użytkownika, co zwiększa pewność siebie chatbota. Jest w sumie siedem
takich zasad nazwanych \verb|laptop r1-7|.


\subsection{Pomodoro}
Ponieważ tematem projektu jest produktywny bot, wymaga on produktywnego systemu
pracy. Jednym z najpopularniejszych i najbardziej efektywnych sytemów jest
właśnie technika Pomodoro. Polega ona na cyklach pracy lub uczenia, gdzie praca
trwa 25 minut, a przerwa 5 minut. Aby chat mierzył czas na żywo, trzeba było
użyć Discorda w połączeniu z Rasą. Do napisania timer'a jest używana funkcja
\verb|start_pomodoro_timer(channel)|. Natomiast połączenie między Rasą a
Discordem polega na wysłaniu wiadomości włączającej pomodor od chatbota, a w
skrypcie Discorda ta wiadomość jest ignorowana, przez co użytkownik jej nie
widzi i następnie jest włączana funkcja obsługująca pomodoro. Pomodoro jest
inicjowane serią pytań, najpierw jest intencja, gdzie użytkownik pyta się czym
jest Pomodoro (\verb|pomodoro|). Następnie chatbot pyta się użytkownika, czy
chcę wypróbować pomodoro, a potem pyta się czy jest gotowy i tutaj są trzy
różne scenariusze:
\begin{itemize}
    \item[\textcolor{violet}{\textbullet}] \verb|Pomodoro timer confirm| - Użytkownik chce wypróbować pomodoro.
    \item[\textcolor{violet}{\textbullet}] \verb|Pomodoro timer deny| - Użytkownik nie chce wypróbowywać
        tej techniki.
    \item[\textcolor{violet}{\textbullet}] \verb|Pomodoro timer confirm not ready| - Użytkownik nie jest
        gotowy.
\end{itemize}
Dodatkowo są tutaj dwie zasady upewniające się, że pomodoro zawsze się zacznie:
\begin{itemize}
    \item[\textcolor{violet}{\textbullet}] \verb|pomodoro r1|
    \item[\textcolor{violet}{\textbullet}] \verb|pomodoro r2|
\end{itemize}











\listoffigures

\end{document}
